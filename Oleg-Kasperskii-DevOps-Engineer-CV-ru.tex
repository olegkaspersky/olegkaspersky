\documentclass[11pt]{article}
\usepackage[letterpaper, top=0.5in, bottom=0.5in, left=0.5in, right=0.5in]{geometry}
\usepackage{XCharter}
\usepackage[T1,T2A]{fontenc}
\usepackage[utf8]{inputenc}
\usepackage[russian]{babel}
\usepackage{enumitem}
\usepackage[hidelinks]{hyperref}
\usepackage{titlesec}
\usepackage{bookmark}
\raggedright
\pagestyle{empty}

\input{glyphtounicode}
\pdfgentounicode=1

\titleformat{\section}{\bfseries\large}{}{0pt}{}[\vspace{1pt}\titlerule\vspace{-6.5pt}]
\renewcommand\labelitemi{$\vcenter{\hbox{\small$\bullet$}}$}
\setlist[itemize]{itemsep=-2pt, leftmargin=12pt, topsep=7pt}

\begin{document}

\centerline{\Huge Олег Касперский, DevOps-инженер}

\vspace{5pt}

\centerline{\href{mailto:oleg.kaspersky0@gmail.com}{oleg.kaspersky0@gmail.com} | \href{https://t.me/oki071}{Telegram} | \href{https://www.linkedin.com/in/oleg-kasperskii/}{LinkedIn} | Белград, Сербия}

\vspace{-10pt}

\section*{О себе}
DevOps-инженер с опытом более 5 лет в CI/CD, Kubernetes, Infrastructure as Code и системах сборки Unreal Engine. Подтверждённый опыт руководства командами, построения платформ наблюдаемости и оптимизации рабочих процессов в средах разработки игр.

\vspace{-6.5pt}

\section*{Навыки}
\textbf{CI/CD:} TeamCity, Docker, Helm, ArgoCD \\
\textbf{Облако и инфраструктура:} GCP, AWS, Kubernetes (GKE), Ansible, Terraform, Terragrunt \\
\textbf{Наблюдаемость:} VictoriaMetrics (Prometheus-совместимая альтернатива), Grafana, Grafana Loki, Grafana Tempo, Vector, OpenTelemetry \\
\textbf{Unreal Engine:} Валидация контента, модификации движка, UGS, Horde, Unsync, UAT, BuildGraph, Cloud DDC, RoboMerge, Zen \\
\textbf{Системы контроля версий:} Git, Perforce \\
\textbf{Программирование:} Python, Unreal Engine C++, C\#, Go \\
\textbf{Архитектура:} Проектирование систем, EventStorming, проектирование микросервисов, планирование проектов, эволюция ПО

\vspace{-6.5pt}

\section*{Опыт работы}
\textbf{Ведущий DevOps-инженер,} GS-Studio (Неанонсированный проект на UE5) -- Белград, Сербия \hfill Июль 2024 -- Настоящее время \\
\vspace{-9pt}
\begin{itemize}
  \item Руководил командой из 3 инженеров; отвечал за планирование, техническое видение и менторство
  \item Внедрил проверки качества для кода и контента; ввёл механизм ratcheting для предотвращения роста предупреждений; сократил предупреждения кода с ~5000 до ~2700 (включая 250 критических); уменьшил предупреждения контента с 250 до 30 (88\%)
  \item Внедрил инструменты экосистемы Unreal Engine: UnrealGameSync, Horde, RoboMerge, BuildGraph, Cloud DDC
  \item Сократил время синхронизации сборок с 15 мин до 1-5 мин (67-93\%) благодаря интеграции Unsync
  \item Уменьшил время синхронизации PCB в UnrealGameSync с 15 мин до 7 мин (53\%) с помощью аналитики Horde и настройки Perforce
  \item Интегрировал GitOps с ArgoCD. Совместно с RenovateBot автоматизировал обновление зависимостей, поддерживая инфраструктуру современной, безопасной и актуальной
  \item Построил стек наблюдаемости (логирование, метрики, трассировка, алертинг); решил критические проблемы k8s API, недоставляемых сообщений и дискового ввода-вывода; обеспечил мониторинг задержки, пропускной способности, насыщения и доступности
  \item Руководил инструментированием серверных сервисов с OpenTelemetry для распределённой трассировки и генерации кастомных метрик
  \item Внедрил безопасность цепочки поставок: SBOM, происхождение артефактов, подписание и верификация подписей
  \item Спроектировал систему развёртывания бэкенд-сервисов для игровых серверов по требованию; сократил затраты на инфраструктуру разработки на 67\% (в 3 раза) и время развёртывания с 10 мин до 30 сек (95\%)
\end{itemize}

\textbf{DevOps-инженер,} Sperasoft (Mortal Kombat Mobile, Injustice 2) -- Белград, Сербия \hfill Июнь 2022 -- Июль 2024 \\
\vspace{-9pt}
\begin{itemize}
  \item Создал и обслуживал CI/CD пайплайны для мобильных проектов на Unreal Engine, SDK: сборка, упаковка, развёртывание
  \item Интегрировал статический анализ и линтинг в CI для контроля качества кода
  \item Перевёл инфраструктуру сборки на IaC (Ansible), устранив дрейф конфигурации
  \item Перевёл конфигурации TeamCity на Configuration as Code (Kotlin DSL), устранив дрейф конфигурации
  \item Сократил время полной синхронизации рабочего пространства Perforce с 50 мин до 12 мин путём настройки прокси Perforce (76\%)
  \item Менторил инженеров, проводил собеседования, внедрял лучшие практики в отделе для Ansible и TeamCity кода
\end{itemize}

\textbf{Системный администратор,} Nival Production -- Санкт-Петербург, Россия \hfill Сентябрь 2019 -- Октябрь 2021 \\
\vspace{-9pt}
\begin{itemize}
  \item Управлял 100+ серверами в средах разработки и стейджинга
  \item Поддерживал TeamCity, GitLab, Atlassian, AD DS, Zabbix с мониторингом и алертингом
  \item Автоматизировал задачи системного администрирования с помощью Ansible, Puppet, Python, PowerShell
\end{itemize}

\end{document}
