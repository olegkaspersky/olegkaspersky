\documentclass[11pt]{article}       % set main text size
\usepackage[letterpaper,                % set paper size to letterpaper. change to a4paper for resumes outside of North America
top=0.5in,                          % specify top page margin
bottom=0.5in,                       % specify bottom page margin
left=0.5in,                         % specify left page margin
right=0.5in]{geometry}              % specify right page margin

\usepackage{XCharter}               % set font. comment this line out if you want to use the default LaTeX font Computer Modern
\usepackage[T1,T2A]{fontenc}        % output encoding (T2A for Cyrillic)
\usepackage[utf8]{inputenc}         % input encoding
\usepackage[russian]{babel}         % Russian language support
\usepackage{enumitem}               % enable lists for bullet points: itemize and \item
\usepackage[hidelinks]{hyperref}    % format hyperlinks
\usepackage{titlesec}               % enable section title customization
\usepackage{bookmark}
\raggedright                        % disable text justification
\pagestyle{empty}                   % disable page numbering

% ensure PDF output will be all-Unicode and machine-readable
\input{glyphtounicode}
\pdfgentounicode=1

% format section headings: bolding, size, white space above and below
\titleformat{\section}{\bfseries\large}{}{0pt}{}[\vspace{1pt}\titlerule\vspace{-6.5pt}]

% format bullet points: size, white space above and below, white space between bullets
\renewcommand\labelitemi{$\vcenter{\hbox{\small$\bullet$}}$}
\setlist[itemize]{itemsep=-2pt, leftmargin=12pt, topsep=7pt}

% resume starts here
\begin{document}

% name
\centerline{\Huge Олег Касперский}

\vspace{5pt}

% contact information
\centerline{\href{mailto:oleg.kaspersky0@gmail.com}{oleg.kaspersky0@gmail.com} | \href{https://t.me/oki071}{Telegram} | \href{https://www.linkedin.com/in/oleg-kaspersky-7887511a7/}{LinkedIn} | \href{https://github.com/olegkaspersky}{GitHub}}

\vspace{-10pt}

% summary section
\section*{О себе}
DevOps-инженер с опытом более 6 лет в CI/CD, Kubernetes, Infrastructure as Code и системах сборки Unreal Engine. Подтверждённый опыт руководства командами, построения платформ наблюдаемости и оптимизации рабочих процессов в средах разработки игр.

\vspace{-6.5pt}

% skills section
\section*{Навыки}
\textbf{CI/CD:} TeamCity, Docker, Helm, ArgoCD \\
\textbf{Облако и инфраструктура:} Google Cloud Platform, Kubernetes, Ansible, Terraform \\
\textbf{Наблюдаемость:} VictoriaMetrics, Grafana, Grafana Loki, Grafana Tempo, Vector, OpenTelemetry \\
\textbf{Программирование:} Python, Unreal Engine C++ и C\#, Go \\
\textbf{Системы контроля версий:} Git, Perforce \\
\textbf{Unreal Engine:} Валидация контента, модификации движка, UnrealGameSync, Horde, Unsync, автоматизация UAT, BuildGraph, Unreal Cloud DDC, RoboMerge \\
\textbf{Архитектура:} Проектирование систем, архитектура решений, проектирование микросервисов, планирование проектов, эволюция ПО

\vspace{-6.5pt}

% experience section
\section*{Опыт работы}
\textbf{Ведущий DevOps-инженер,} GS-Studio -- Удалённо \hfill Июль 2024 -- Настоящее время \\
\vspace{-9pt}
\begin{itemize}
  \item Руководил командой из 3 DevOps-инженеров; отвечал за планирование, техническое видение и менторство
  \item Сократил время синхронизации сборок с 15 мин до 1-5 мин благодаря интеграции Unsync
  \item Уменьшил время синхронизации PCB в UnrealGameSync с 15 мин до 7 мин с помощью аналитики Horde и настройки Perforce
  \item Спроектировал инфраструктуру по разворачиванию инфраструктуры для игровых серверов по требованию; сократил затраты на инфраструктуру разработки в 3 раза и время развёртывания бэкенда с 10 мин до 30 сек
  \item Построил стек наблюдаемости (логирование, метрики, трассировка, алертинг); решил критические проблемы k8s API, недоставляемых сообщений и дискового ввода-вывода
  \item Руководил инструментированием серверных сервисов с OpenTelemetry для распределённой трассировки и генерации кастомных метрик
  \item Внедрил проверки для кода и контента; исправив 4000+ проблем в коде, включая 250 критических, снизив количество предупреждений по контенту с 250 до 30
  \item Внедрил безопасность цепочки поставок: SBOM, происхождение артефактов, подписание и верификация подписей
  \item Внедрил инструменты экосистемы Unreal Engine: UnrealGameSync, Horde, RoboMerge, BuildGraph, Cloud DDC
  \item Автоматизировал обновление сервисов размещенных в Kubernetes с помощью RenovateBot и ArgoCD
\end{itemize}

\textbf{DevOps-инженер,} Sperasoft, Keywords D.O.O -- Белград, Сербия \hfill Июнь 2022 -- Июль 2024 \\
\vspace{-9pt}
\begin{itemize}
  \item Создал и обслуживал CI/CD пайплайны для мобильных проектов на Unreal Engine: сборка, упаковка, развёртывание
  \item Интегрировал статический анализ и линтинг в CI для контроля качества кода
  \item Перевёл инфраструктуру на IaC (Ansible), устранив дрейф конфигурации
  \item Перевёл TeamCity на Configuration as Code (Kotlin DSL)
  \item Сократил время полной синхронизации рабочего пространства Perforce с 50 мин до 12 мин
  \item Создал кастомные уведомления о сборках для снижения усталости команды от алертов
  \item Менторил инженеров, проводил собеседования, внедрял лучшие практики
\end{itemize}

\textbf{Системный администратор,} Nival Production -- Санкт-Петербург, Россия \hfill Сентябрь 2019 -- Октябрь 2021 \\
\vspace{-9pt}
\begin{itemize}
  \item Управлял 100+ серверами в средах разработки и стейджинга
  \item Поддерживал TeamCity, GitLab, Atlassian, AD DS, Zabbix с мониторингом и алертингом
  \item Автоматизировал задачи системного администрирования с помощью Ansible, Puppet, Python, PowerShell
\end{itemize}

\vspace{-18.5pt}

\end{document}
